\section{Modelagem em grafos}

Para a modelagem do problema, considere os vértices como representações das variáveis do programa,
e as arestas indicam conflitos entre variáveis — isto é, situações em que duas variáveis estão ativas
ao mesmo tempo e, portanto, não podem compartilhar o mesmo registrador. O grafo formado por essa relação
é chamado de \textit{grafo de interferência}.

Dessa forma, uma coloração adequada do grafo de interferência — atribuindo uma cor diferente
para cada conjunto de variáveis que não podem ser simultâneas — corresponde a uma alocação de registradores sem conflitos.

Se a coloração do grafo puder ser feita com um número de cores menor ou igual ao número de registradores disponíveis,
então é possível realizar a alocação diretamente. Caso contrário, é necessário utilizar alguma técnica de spilling para
lidar com as variáveis que não puderem ser atribuidas a um registrador.
