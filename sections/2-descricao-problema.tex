\section{Descrição do problema}

O problema de alocação de registradores consiste em atribuir variáveis a um número limitado de registradores durante a compilação de um programa.

\subsection{Objetivo do problema}

O objetivo é estabelecer uma associação entre as variáveis do programa e os registradores disponíveis,
determinando, sempre que possível, a qual registrador cada variável será atribuída.

Uma solução considerada ótima é aquela que utiliza o menor número possível de registradores para
alocar todas as variáveis do programa.

Por exemplo, considere um programa com cinco variáveis. Em uma primeira tentativa de alocação, cada variável é atribuída
a um registrador diferente, totalizando cinco registradores utilizados. Essa é uma solução válida.
No entanto, se for possível alocar todas as cinco variáveis utilizando apenas três registradores,
essa será uma solução melhor. Além disso, se o número disponível de registradores for de fato três,
essa alocação representa uma solução ótima, pois atende à limitação de recursos.

Em certos casos, pode não existir uma solução válida com o número disponível de registradores, ou seja, não é possível
atribuir cada variável a um registrador sem conflitos. Nesses cenários, a literatura propõe abordagens alternativas,
como a técnica de \textit{spilling}, que transfere algumas variáveis da memória rápida (registradores) para
a memória principal (RAM) durante a execução.

\subsection{Entradas esperadas}

A entrada do problema consiste em uma descrição das variáveis do programa e suas interações,
indicando quais variáveis estão ativas ao mesmo tempo e, portanto, não podem compartilhar o mesmo registrador.

Essa descrição pode assumir diferentes formas, como uma tabela de intervalos de uso de variáveis (lifetimes),
uma matriz de interferência, ou qualquer outro formato que permita inferir conflitos de uso simultâneo entre variáveis.

\subsection{Saídas esperadas}

A saída deve indicar se existe uma atribuição válida das variáveis aos registradores disponíveis.
Caso exista, a solução deve apresentar a alocação realizada, especificando a qual registrador cada variável foi atribuída.

Em caso de impossibilidade (isto é, se não for possível realizar a alocação com o número de registradores fornecido),
a saída deve indicar que não há solução válida. Alternativamente, pode apresentar uma solução com \textit{spilling},
identificando quais variáveis foram movidas para a memória.
