\documentclass[a4paper,12pt]{article}
\usepackage[utf8]{inputenc}
\usepackage[brazil]{babel}
\usepackage{booktabs}
\usepackage{caption}
\usepackage[a4paper,top=2cm]{geometry}
\usepackage{graphicx}
\usepackage{amsmath, amssymb, amsthm}
\usepackage{listings}
\usepackage{xcolor} 
\usepackage{enumitem}
\usepackage{csquotes}
\usepackage[backend=biber,style=authoryear]{biblatex}
\usepackage{algorithm}
\usepackage{algorithmic}

\addbibresource{referencias.bib}

\lstset{
    basicstyle=\ttfamily\small,
    keywordstyle=\color{blue},
    stringstyle=\color{red},
    commentstyle=\color{gray},
    numbers=left,
    numberstyle=\tiny\color{gray},
    breaklines=true,
    tabsize=4,
    showstringspaces=false,
}


\title{Estado da arte de alocação de registradores}

\author{Raoni Silva, Raul Ramalho}
\date{\today}

\begin{document}

\maketitle

\noindent Turma de Grafos 35M34 \\ Unidade 2

\newpage

\tableofcontents


\section{Resumo}


\section{Introdução}

Registradores são componentes de memória de uso genérico de uma CPU de um computador.
Sua função é de armazenar dados e intruções que serão processados de imediato. É a tecnologia de maior nível na hierárquia de memória em um computador.

Pela sua quantidade reduzida, uso intermitente e alta importância, sua ociosidade ou desperdício não é desejável.
O uso ótimo destes componentes é portante prioritário. Para isso, pesquisa e técnicas de otimização foram desenvolvidas nesta área.

Entre as diversas abordagens disponíveis, uma delas adotadas fora a de coloração de grafos de interferência. Técnica que consiste
em cada variável em dado programa de computador é representada por um vértice no grafo, e suas arestas representam a coexistência
do tempo de vida desta variável no mesmo instante. Cada cor atribuída ao grafo corresponde ao número de registradores de propósito geral disponíveis.

Inicialmente, em 1981 nos laboratórios da IBM, o cientista Chaitin e seus colegas de pesquisa desenvolveram o primeiro algoritmo otimização
do uso de registradores com coloração de grafos de interferência.

O algoritmo é especificado em quatro etapas, construção do grafo de interferência, armazenamento na memória e coloração, spilling
( armazenamento na ram) e recuperação das variáveis na ram e recosntrução do grafo.

Posteriormente, o cientista \textcite{briggs} e seus colegas da Rice University, interessados no problema decidem refinar o
algoritmo de \textcite{chaitin} e companhia, resultando na aprimoração das heurísticas na etapa de coalescência, simplificação e spilling(derramamento).

Por fim, em 1996, pelo Bell Labs na equipe de George e Appel et al., novamente, a técnica de coloração de grafos de interferência
fora otimizada mais uma vez apartir de Briggs e companhia. Agora, a técnica de coalescência admite uma nova heurística, critérios
mais robustos foram atribuidas como prevenção do pior caso do algoritmo de seus antecessores. Uma nova etapa de congelamento
fora adiciona antes de ocorrer o spilling. Atingindo, assim, o estado da arte.

\section{Descrição do problema}

O problema de alocação de registradores consiste em atribuir variáveis a um número limitado de registradores durante a compilação de um programa.

\subsection{Objetivo do problema}

O objetivo é estabelecer uma associação entre as variáveis do programa e os registradores disponíveis,
determinando, sempre que possível, a qual registrador cada variável será atribuída.

Uma solução considerada ótima é aquela que utiliza o menor número possível de registradores para
alocar todas as variáveis do programa.

Por exemplo, considere um programa com cinco variáveis. Em uma primeira tentativa de alocação, cada variável é atribuída
a um registrador diferente, totalizando cinco registradores utilizados. Essa é uma solução válida.
No entanto, se for possível alocar todas as cinco variáveis utilizando apenas três registradores,
essa será uma solução melhor. Além disso, se o número disponível de registradores for de fato três,
essa alocação representa uma solução ótima, pois atende à limitação de recursos.

Em certos casos, pode não existir uma solução válida com o número disponível de registradores, ou seja, não é possível
atribuir cada variável a um registrador sem conflitos. Nesses cenários, a literatura propõe abordagens alternativas,
como a técnica de \textit{spilling}, que transfere algumas variáveis da memória rápida (registradores) para
a memória principal (RAM) durante a execução.

\subsection{Entradas esperadas}


A entrada do problema consiste em uma descrição das variáveis do programa e suas interações,
indicando quais variáveis estão ativas ao mesmo tempo e, portanto, não podem compartilhar o mesmo registrador.

Essa descrição pode assumir diferentes formas, como uma tabela de intervalos de uso de variáveis (lifetimes),
uma matriz de interferência, ou qualquer outro formato que permita inferir conflitos de uso simultâneo entre variáveis.

\subsection{Saídas esperadas}

A saída deve indicar se existe uma atribuição válida das variáveis aos registradores disponíveis.
Caso exista, a solução deve apresentar a alocação realizada, especificando a qual registrador cada variável foi atribuída.

Em caso de impossibilidade (isto é, se não for possível realizar a alocação com o número de registradores fornecido),
a saída deve indicar que não há solução válida. Alternativamente, pode apresentar uma solução com \textit{spilling},
identificando quais variáveis foram movidas para a memória.

\section{Modelagem em grafos}

Para a modelagem do problema, considere os vértices como representações das variáveis do programa,
e as arestas indicam conflitos entre variáveis — isto é, situações em que duas variáveis estão ativas
ao mesmo tempo e, portanto, não podem compartilhar o mesmo registrador. O grafo formado por essa relação
é chamado de \textit{grafo de interferência}.

Dessa forma, uma coloração adequada do grafo de interferência — atribuindo uma cor diferente
para cada conjunto de variáveis que não podem ser simultâneas — corresponde a uma alocação de registradores sem conflitos.

Se a coloração do grafo puder ser feita com um número de cores menor ou igual ao número de registradores disponíveis,
então é possível realizar a alocação diretamente. Caso contrário, é necessário utilizar alguma técnica de spilling para
lidar com as variáveis que não puderem ser atribuidas a um registrador.

\section{Estado da arte}



\section{Revisão bibliográfica}

\section{Casos de teste}

A principal comparação analisada neste trabalho é entre os algoritmos de Briggs e George-Appel.
Como o algoritmo de George-Appel representa uma evolução direta sobre o de Briggs, essa comparação é
especialmente relevante para evidenciar as melhorias introduzidas.

No estudo apresentado por \textcite{irc}, é realizada uma comparação empírica entre os algoritmos utilizando sete programas escritos na
linguagem Standard ML, compilados com o compilador SML/NJ. Esses programas foram selecionados por possuírem características variadas,
tais como: diferentes tamanhos, presença ou ausência de modularização, e variáveis com durações de vida longas ou curtas.
Essas variações tornam o conjunto de testes representativo de diferentes cenários encontrados em compiladores reais, conferindo robustez
à comparação entre os algoritmos.

Uma limitação importante dos algoritmos originais de Briggs, identificada no estudo, é a ausência de suporte adequado para vértices
pré-coloridos — situação comum em chamadas de função ou em otimizações específicas do compilador. Devido ao uso de estratégias de coalescência
mais agressivas, esses algoritmos, por vezes, geravam grafos não coloríveis, impedindo a finalização do processo de alocação de registradores.
Assim, a comparação entre os algoritmos foi conduzida apenas nas situações em que o algoritmo de Briggs produzia resultados viáveis, de modo
a garantir uma análise justa e tecnicamente válida.

\subsection{Expectativas}

Visto que a melhoria do algoritmo de George-Appel é focalizada no processo de coalescência, é esperado que as melhorias se encontrem nessa parte, ou seja
a quantidade de nós agrupados seja maior do que no algoritmo de Briggs. Dessa forma, reduzindo a quantidade de instruções de move 
($x \leftarrow y$) no código de máquina gerado.


\subsection{Expectativas}

Como a principal melhoria introduzida pelo algoritmo de George-Appel está relacionada ao processo de coalescência, é natural que os ganhos
de desempenho se manifestem nessa etapa. Especificamente, espera-se um aumento na quantidade de nós coalescidos, ou seja, uma maior fusão de
movimentos(instruções de move, do tipo $x \leftarrow y$) entre variáveis , o que leva à geração de menos instruções de transferência no código 
de máquina final. Essa otimização contribui tanto para a redução do tamanho do código quanto para sua eficiência em tempo de execução.

\subsection{Resultados}

\subsubsection{Coalescência}

No primeiro conjunto de testes, a métrica avaliada foi a taxa de vértices coalescidos no grafo de interferência. O algoritmo de Briggs
apresentou uma média de aproximadamente 64\% de coalescência, enquanto o algoritmo de George-Appel atingiu cerca de 86\%. Esse aumento significativo
confirma as expectativas iniciais e evidencia a eficácia da abordagem aprimorada. A maior taxa de coalescência implica na eliminação de mais instruções
de movimento, resultando em um código de máquina mais enxuto e eficiente.

\subsubsection{Tamanho do código}

Como consequência direta da coalescência aprimorada, observou-se uma redução no tamanho do código assembly gerado. O algoritmo de George-Appel produziu,
em média, um código 5\% menor em relação ao gerado pelo algoritmo de Briggs. Essa diferença, embora modesta, pode representar ganhos importantes
em ambientes com restrições de memória ou largura de banda.

\subsubsection{Desempenho em tempo de execução}

Por fim, as melhorias introduzidas no processo de alocação impactaram também o desempenho em tempo de execução dos programas compilados.
As otimizações resultaram em uma redução média de 4,4\% no tempo de execução dos casos de teste. Essa melhora reforça o argumento de que estratégias
de coalescência mais eficazes não apenas reduzem o número de instruções, como também contribuem para a geração de código mais rápido.

\section{Proposta de abordagem}

\section{Conclusão}

Neste trabalho, investigou-se a alocação de registradores via coloração de grafos, comparando
implementações dos algoritmos de \textcite{chaitin}, \textcite{briggs} e \textcite{irc}.  
Observou-se que Chaitin apresenta implementação mais simples, mas tende a maior número
de spills em relação a abordagens refinadas.  
Briggs introduz heurísticas de coalescing otimista que reduzem moves, porém aumentam
a complexidade de análise.  
George-Appel (Iterated Register Coalescing) demonstrou maior taxa de coalescência e menos spills,
sendo uma melhoria direta em relação do de Briggs.  
Em suma, refinamentos em coalescing e heurísticas de spill podem melhorar significativamente a geração de código
sem comprometer o tempo de compilação.

\section{Referências}

aula 10 de grafos (aula de coloração)

https://web.eecs.umich.edu/~mahlke/courses/583f12/reading/chaitin82.pdf



\end{document}
